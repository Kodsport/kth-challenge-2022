\problemname{Box and Arrow Diagram}

\illustration{0.42}{box.png}{An example of a box and arrow diagram, taken from github.com/dicander/box\_arrow\_diagram}
What an embarrassment! Itaf got 0/5 points in her last ``Fundamental programming in Python'' exam. 
She studies Engineering physics at KTH and is struggling with this course. She is not alone, 
as $60\%$ of her classmates failed the exam this year. The reason for this oddly high percentage 
is the so called \textit{box and arrow diagram} (låd- och pildiagram).

In this part of the exam you are given a piece of Python code and you have to draw how the memory 
structure will look like when the program reaches a given line. Since Itaf is a high-rated 
competitive programmer her ego always came in the way whenever she tried to study for the test, 
because it felt ``too easy''. But now she has become desperate and needs your help.

The \textit{box and arrow diagram} is used to explain the memory structure inside Python. 
Simplified, the diagram can be seen as a directed graph with nodes (boxes) labeled from $1$ to $N$ and 
edges (arrows) labeled from $1$ to $M$. The boxes corresponds to the objects in the memory
of a Python program. Box 1 is special, it represents the \textit{global} object.  
An arrow being drawn from box $u$ to box $v$ in the diagram means that object $u$ stores a reference 
of object $v$. If $u$ stores multiple references of $v$, then you draw multiple arrows 
from $u$ to $v$. It is also possible for an object to contain references to itself.

An object $u$ is said to be \textit{alive} if there exists a path from the \textit{global} 
object to $u$ in the \textit{box and arrow diagram}. Each object also has a reference counter. 
The reference counter of an object $u$ is defined as the number of arrows $(v,u)$ such that $v$ is alive. 

Itaf now needs your help, and she will ask you $Q$ queries, each query can be one of two types.

\begin{itemize}
	\item{\texttt{1 X} \, Remove the arrow with label $X$ from the diagram.}
	\item{\texttt{2 Y} \, Output the reference counter of the object with label $Y$.}
\end{itemize}

\section*{Input}
The first line consists of two space separated integers $N,M$ ($1 \leq N,M \leq 2 \cdot 10^5$), 
where $N$ is the number of boxes in the diagram and $M$ is the number of arrows in the diagram. 

The next $M$ lines describe the arrows in the diagram. The $i$-th line contains $2$ space separated integers $U_i,V_i$ ($1 \leq U_i,V_i \leq N$), 
meaning the arrow with label $i$ goes from box $U_i$ to box $V_i$. 
Note that arrows forming loops and multi-edges are allowed.

The next line contains an integer $Q$ ($1 \leq Q \leq 2 \cdot 10^5$), the number of queries.
The next $Q$ lines describe the $Q$ queries. The $j$-th query is given as a pair of space separated integers $C_j, X_j$ ($1 \leq C_j \leq 2$).

\begin{itemize}
 \item If $C_j = 1$ then remove the arrow labeled $X_j$ from the diagram ($1 \leq X_j \leq M$).
 \item If $C_j = 2$ then output the reference counter of object $X_j$ ($1 \leq X_j \leq N$). 
\end{itemize}
It is guaranteed that there will not be two queries of type $1$ with same value of $X_j$,
meaning the same arrow will never be deleted twice.

\section*{Output}
For each query of type $2$, output a single line containing the reference count of object $Y_j$.
